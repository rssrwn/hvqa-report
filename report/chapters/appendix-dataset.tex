\documentclass[../report.tex]{subfiles}


\begin{document}


\chapter{OceanQA Examples}
\label{appendix-dataset}

This supplementary material provides a number of example questions and video frames from the OceanQA dataset. The appendix is split into sections based on the question type. Each section outlines a number of example questions and some of the video frames which relate to each question.


\vspace{1cm}
{\large \textbf{Property Questions}}

Q: What rotation was the brown rock in frame 21? \\A: upward-facing

Q: What colour was the right-facing fish in frame 27? \\A: silver

Figure~\ref{fig:prop-qs-frame-examples} shows frames 21 and 27 from these videos.

\begin{figure}[h!]
  \centering
  \begin{subfigure}{0.4\textwidth}
    \centering
    \includegraphics[width=\textwidth]{prop-example-21.png}
    \caption{Frame 21.}
  \end{subfigure}
  \begin{subfigure}{0.4\textwidth}
    \centering
    \includegraphics[width=\textwidth]{prop-example-27.png}
    \caption{Frame 27.}
  \end{subfigure}
  \caption{ }
  \label{fig:prop-qs-frame-examples}
\end{figure}

\pagebreak

{\large \textbf{Relation Questions}}

Q: Was the purple rock close to the octopus in frame 7? \\A: yes

Q: Was the upward-facing fish close to the brown rock in frame 20? \\A: no

Figure~\ref{fig:relation-examples} shows the two frames these questions are referring to.

\begin{figure}[h!]
  \centering
  \begin{subfigure}{0.4\textwidth}
    \centering
    \includegraphics[width=\textwidth]{close-example-yes.png}
    \caption{Frame 7.}
  \end{subfigure}
  \begin{subfigure}{0.4\textwidth}
    \centering
    \includegraphics[width=\textwidth]{close-example-no.png}
    \caption{Frame 20.}
  \end{subfigure}
  \caption{ }
  \label{fig:relation-examples}
\end{figure}


\vspace{1cm}
{\large \textbf{Action Questions}}

Q: Which action occurred immediately after frame 12? \\A: move

Q: Which action occurred immediately after frame 16? \\ A: rotate anti-clockwise

Figure~\ref{fig:cp-example-rotation} gives an example of an octopus rotating anti-clockwise.


\vspace{1cm}
{\large \textbf{Changed-Property Questions}}

Q: What happened to the octopus immediately after frame 0? \\A: Its rotation changed from upward-facing to left-facing

Figure~\ref{fig:cp-example-rotation} shows the first and second frame from this video.

Q: What happened to the octopus immediately after frame 20? \\A: Its colour changed from red to blue

Figure~\ref{fig:cp-example-cc} shows frame 20 and frame 21 from this video.

\begin{figure}[h!]
  \centering
  \begin{subfigure}{0.4\textwidth}
    \centering
    \includegraphics[width=\textwidth]{cp-example-rot-0.png}
  \end{subfigure}
  \begin{subfigure}{0.4\textwidth}
    \centering
    \includegraphics[width=\textwidth]{cp-example-rot-1.png}
  \end{subfigure}
  \caption{An example of an octopus rotating clockwise.}
  \label{fig:cp-example-rotation}
\end{figure}

\begin{figure}[h!]
  \centering
  \begin{subfigure}{0.4\textwidth}
    \centering
    \includegraphics[width=\textwidth]{cp-example-cc-0.png}
  \end{subfigure}
  \begin{subfigure}{0.4\textwidth}
    \centering
    \includegraphics[width=\textwidth]{cp-example-cc-1.png}
  \end{subfigure}
  \caption{An example of an octopus changing colour.}
  \label{fig:cp-example-cc}
\end{figure}

\pagebreak


{\large \textbf{Repetition Count Questions}}

Q: How many times does the octopus rotate clockwise? \\A: 0

Figure~\ref{fig:st-example-rc} gives an example of an octopus rotating clockwise and then moving.

Q: How many times does the octopus eat a bag? \\A: 1

Figure~\ref{fig:rc-example-eab} shows the octopus eating a bag.

\begin{figure}[h!]
  \centering
  \begin{subfigure}{0.32\textwidth}
    \centering
    \includegraphics[width=\textwidth]{rc-example-eab-0.png}
  \end{subfigure}
  \begin{subfigure}{0.32\textwidth}
    \centering
    \includegraphics[width=\textwidth]{rc-example-eab-1.png}
  \end{subfigure}
  \begin{subfigure}{0.32\textwidth}
    \centering
    \includegraphics[width=\textwidth]{rc-example-eab-2.png}
  \end{subfigure}
  \caption{An example of an octopus eating a bag.}
  \label{fig:rc-example-eab}
\end{figure}


\vspace{1cm}
{\large \textbf{Repeating Action Questions}}

Q: What does the octopus do 1 times? \\A: eat a fish

Figure~\ref{fig:ra-example-eaf} shows an octopus eating a fish.

Q: What does the octopus do 4 times? \\A: rotate clockwise

\begin{figure}[h!]
  \centering
  \begin{subfigure}{0.32\textwidth}
    \centering
    \includegraphics[width=\textwidth]{ra-example-0.png}
  \end{subfigure}
  \begin{subfigure}{0.32\textwidth}
    \centering
    \includegraphics[width=\textwidth]{ra-example-1.png}
  \end{subfigure}
  \begin{subfigure}{0.32\textwidth}
    \centering
    \includegraphics[width=\textwidth]{ra-example-2.png}
  \end{subfigure}
  \caption{An example of the octopus eating a fish.}
  \label{fig:ra-example-eaf}
\end{figure}


\vspace{1cm}
{\large \textbf{State Transition Questions}}

Q: What does the octopus do immediately after rotating clockwise for the fourth time? \\A: move

Figure~\ref{fig:st-example-rc} shows the video frames which this question is asking about.

% Another example

\begin{figure}[h!]
  \centering
  \begin{subfigure}{0.32\textwidth}
    \centering
    \includegraphics[width=\textwidth]{st-example-rc-0.png}
  \end{subfigure}
  \begin{subfigure}{0.32\textwidth}
    \centering
    \includegraphics[width=\textwidth]{st-example-rc-1.png}
  \end{subfigure}
  \begin{subfigure}{0.32\textwidth}
    \centering
    \includegraphics[width=\textwidth]{st-example-rc-2.png}
  \end{subfigure}
  \caption{An example of an octopus rotating clockwise before moving.}
  \label{fig:st-example-rc}
\end{figure}



\end{document}
